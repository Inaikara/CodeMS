\documentclass[journal]{IEEEtran}
\usepackage{amsmath,amsfonts}
\usepackage{algorithmic}
\usepackage{algorithm}
\usepackage{array}
\usepackage[caption=false,font=normalsize,labelfont=sf,textfont=sf]{subfig}
\usepackage{textcomp}
\usepackage{stfloats}
\usepackage{url}
\usepackage{verbatim}
\usepackage{graphicx}
\usepackage{cite}
\hyphenation{op-tical net-works semi-conduc-tor IEEE-Xplore}
% updated with editorial comments 8/9/2021

\begin{document}
\title{Title}
\author{Wenjin Xu, Chenguang Yang~\IEEEmembership{Fellow,~IEEE,}
    \thanks{This paper was produced by the IEEE Publication Technology Group. They are in Piscataway, NJ.}
    \thanks{Manuscript received April 19, 2021; revised August 16, 2021.}}

% The paper headers
\markboth{Journal of \LaTeX\ Class Files,~Vol.~14, No.~8, August~2021}%
{Shell \MakeLowercase{\textit{et al.}}: A Sample Article Using IEEEtran.cls for IEEE Journals}

\IEEEpubid{0000--0000/00\$00.00~\copyright~2021 IEEE}
% Remember, if you use this you must call \IEEEpubidadjcol in the second
% column for its text to clear the IEEEpubid mark.

\maketitle

\begin{abstract}

\end{abstract}

\begin{IEEEkeywords}

\end{IEEEkeywords}

\section{Introduction}
\IEEEPARstart{T}{he} main contributions of this article are listed as follows. \cite{Ju2012}.

1) We propose a learning method based on DMP and FGMM which can not only learn the trajectory from multiple demonstrations but also learn the deviation among them.

2) We present a new LfD method which can demonstrate by single static image, each sample point on the image is encoded by a prior trajectory and FGMM. Skills are then learned through the proposed methods

3) We give a method for generating a priori trajectories of Chinese characters, and design a robot learning framework for learning to write. Experiments were carried out on the LASA dataset and Chinese character images to verify the effectiveness of the method.

\section{Learning the trajectory and deviation \\from multiple demonstrations}
Here
\subsection{Fuzzy Gaussian Mixture Model}



\subsection{Dynamic Motion Primitives}

\section{Learning from static image}


\section{Experiments}
\subsection{Robot writing learning framework}

\section{Conclusion}



\bibliographystyle{IEEEtran}
\bibliography{IEEEabrv,reference.bib}

\end{document}


